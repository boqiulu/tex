\documentclass{beamer}

% For more themes, color themes and font themes, see:
% http://deic.uab.es/~iblanes/beamer_gallery/index_by_theme.html
%
\mode<presentation>
{
  \usetheme{CambridgeUS}       % or try default, Darmstadt, Warsaw, ...
  \usecolortheme{seagull} % or try albatross, beaver, crane, ...
  \usefonttheme{professionalfonts}    % or try default, structurebold, ...
  \setbeamertemplate{navigation symbols}{}
  \setbeamertemplate{caption}[numbered]
  \setbeamertemplate{caption}{\insertcaption}
} 

\newcommand{\ditto}[1][.4pt]{\xrfill{#1}~\textquotedbl~\xrfill{#1}}

\usepackage[english]{babel}
\usepackage[utf8x]{inputenc}
\usepackage{graphicx}
\usepackage{booktabs}
\usepackage{paralist}
\usepackage{amsmath}
\usepackage{color}
\usepackage{ragged2e}





\AtBeginSection[]
{
  \begin{frame}
    \frametitle{Table of Contents}
    \tableofcontents[currentsection]
  \end{frame}
}


% On Overleaf, these lines give you sharper preview images.
% You might want to `comment them out before you export, though.
\usepackage{pgfpages}

\pgfpagesuselayout{resize to}[letterpaper, landscape] 


% Here's where the presentation starts, with the info for the title slide

\title[{\tt GaaOptions}]{{\tt GaaOptions}: International Index Options and Enhancements to the Calendar Collar}
\author[DC $|$ BL]{Dimitri Curtil  \and Boqiu Lu}
\date{\today}

\begin{document}

\begin{frame}
  \titlepage
\end{frame}

\begin{frame}{Agenda}
\begin{itemize}
\item Reminder
	\begin{itemize}
	\item Goal of this project so far is to build a better risk-reward payoff for strategically important equity exposures
	\item Accordingly, our bias is to avoid introducing outright model risk that produces a good (single asset price path) backtest but very likely fails in future
	\end{itemize}
\item International Options Overview
	\begin{itemize}
	\item Data: history, density, vol surface
	\end{itemize}
\item Calendar collars for select international options: Eurostoxx, FTSE, Nikkei
\item Enhancements to Calendar Collars
	\begin{itemize}
	\item Replace 20\% OTM with ATM, at a lower notional
	\item Collar implemented with $\Delta$'s instead of moneyness
	\item Collars with even shorter maturity calls
	\item Summarizing enhancements to the S\&P500 Calendar Collar
%	\item TBD: Enhancing the roll schedule with a premium recovery threshold
%	\item TBD: Examining Collar P/L sensitivity via fuzzy properties for options (delta, moneyness, maturity)
	\end{itemize}

\item Next Steps

	\begin{itemize}
	\item Optimizing Calendar Collar Structures for International Options
	\item Other structures, timing etc.
	\end{itemize}

\end{itemize}
\end{frame}

\section{International Options \& Enhancements to {\tt GaaOptions}}
\begin{frame}{Enhancements to {\tt GaaOptions}}
\begin{itemize}
\item Refactored the framework to work with {\em any security} in the Ivy Options DB
\begin{itemize} \item includes ETFs (eg IEF/TLT) or single name securities (eg AMZN) or any international indices (eg NKY, KOSPI etc) \end{itemize}
\item Unified access layer for different databases (AS, US, EU) which are largely similar but have significant differences in schema \& content
\item Significantly improved the performance of the back testing framework
\begin{itemize} \item this enhancement allowed us to {\em include weekly index options} in the options surface \end{itemize}
\item Ability to run the framework with ``fuzzy'' properties for delta, moneyness, maturity etc. 
\item The framework is more robust now and is better equipped to deal with the imperfections and other contours of the IvyDB data
\end{itemize}
\end{frame}

\begin{frame}{Global Options Data}
\scalebox{0.62}{
\begin{tabular}{llllll}
\toprule
Ticker & Issuer & From & Periodicity & Primary Exchange (IvyDB data) & Contract Size\\
\midrule
SPX & S\&P 500 & 1996-01-04 & W, M, Q & Chicago Board Options Exchange & 100\\
\addlinespace
HSI & HANG SENG & 2006-01-03 & M & Hong Kong Futures Exchange & 50\\
HSCEI & HS CHINA ENT INDEX & 2006-01-03 & M & Hong Kong Futures Exchange & 50\\
KOSPI2 & KOSPI 200 INDEX & 2004-05-03 & M & Korea Futures Market & 250000\\
NKY & NIKKEI 225 & 2004-05-06 & W, M & Osaka Day Session & 1000\\
AS51 & S \& P/ASX 200 & 2004-01-02 & W, M & Australia Futures and Options & 10\\
TWSE & TAIEX & 2004-01-02 & W, M & Taiwan Futures Exchange & 50\\
TPX & TOPIX INDEX SEC 1 & 2006-01-04 & M & Osaka Day Session & 10000\\
\addlinespace
AEX & AEX INDEX & 2002-01-02 & W, M & Euronext Amsterdam Options & 100\\
CAC & CAC 40 & 2003-04-14 & W, M & Euronext MONEP & 10\\
DAX & DAX IND & 2002-01-02 & W, M & EUREX, Frankfurt & 5\\
SX5E & DJ EURO STOXX50 & 2002-01-02 & W, M & EUREX, Frankfurt & 10\\
DJGTE & DJ GL TIT 50 EUR IN & 2002-01-02 & & EUREX, Frankfurt\\
SX5P & DJ STOXX 50 & 2002-01-02 & M & EUREX, Frankfurt & 10\\
SXXP & DJ STOXX 600 & 2005-09-14 & M & EUREX, Frankfurt & 10\\
BEL20 & EURONEXT BEL-20 & 2002-01-02 & & Euronext BELFOX, Brussels\\
UKX & FTSE 100 & 2002-01-02 & W, M & Euronext Liffe, London & 10\\
MIB & FTSE MIB & 2006-10-10 & W, M & Mercato dei Derivati, Milano & 2.5\\
IBEX & IBEX-35 & 2006-10-11 & Mini Fut Opt & Mercado Espa{\~n}ol de Futuros & 1\\
OMXH25 & OMXH HELSINKI 25 & 2002-01-02 & M & EUREX, Frankfurt & 10\\
OMXS30 & OMXS30 & 2006-12-21 & W, M & Stockholmborsen Options Market & 100\\
SMI & SMI & 2002-01-02 & W, M & EUREX, Frankfurt & 10\\
V2TX & VSTOXX IND & 2010-03-17 & M & EUREX, Frankfurt & 100\\
\bottomrule
\end{tabular}
}
\end{frame}

\begin{frame}{Several issues with Ivy DB for International Options}
\begin{itemize}
\item Multiple prices can exist for each index/date---depending on which exchange it is traded on
\item IvyDB AS/EU has a `bug' in its Total Return data---basically, they are just price returns! We are currently patching it with total returns and prices from Bloomberg
\item IvyDB AS/EU data have a reference exchange (used to compute Greeks) which may/not be the exchange the options actually trade on and these can be time varying \footnote{$\Delta$'s are modeled, and can be missing on occasion because the model couldn't compute it}
\item Lots of `holes' - prices can go missing for extended periods (eg up to a month), or they can just disappear before expiry
\item {\bf NO bid-offer spreads}
\end{itemize}
\end{frame}

\begin{frame}{Initial Look at the Data - Options Liquidity (Europe), LTM}

\begin{itemize}
\item For Europe, liquidity is most available in Euro Stoxx 50 (SX5E) followed by DAX \& UKX; perhaps additional OTC volume available?
\end{itemize}

\includegraphics[scale=0.35]{liquidity_EU}

\end{frame}


\begin{frame}{Initial Look at the Data - Options Liquidity (Asia), LTM}


\begin{itemize}
\item For Asia, NKY offers most liquidity (TPX no volume) followed by TWSE, HSI, AS51 (KOSPI2 to be verified / to follow)
\end{itemize}

\includegraphics[scale=0.35]{liquidity_AS}

\end{frame}


\begin{frame}{Initial Look at the Data - Options Liquidity (S\&P 500)}


\begin{itemize}
\item S\&P 500 index options liquidity is still significantly larger than all other international index options markets combined
\end{itemize}

\includegraphics[scale=0.35]{liquidity_US}

\end{frame}

\begin{frame}{Initial Look at the Data - Options Availability (S\&P)}


\begin{itemize}
\item S\&P offers a range of listed options (weekly, monthly, quarterly) that extends far OTM and to maturities that are nearly 2.5 years out.
\end{itemize}

\includegraphics[scale=0.35]{moneyness_maturity_US}

\end{frame}


\begin{frame}{Initial Look at the Data - Options Availability (Asia)}


\begin{itemize}
\item For Asian indices, moneyness does not extend as wide of range, but maturities can available for out to 5 years (TPX, NKY)
\end{itemize}

\includegraphics[scale=0.35]{moneyness_maturity_AS}

\end{frame}

\begin{frame}{Initial Look at the Data - Options Availability (Europe)}


\begin{itemize}
\item For European indices, listed maturities can extend out to 10 years with strikes ranging from 80\% OTM and ITM (SX5E, UKX)
\end{itemize}

\includegraphics[scale=0.35]{moneyness_maturity_EU}

\end{frame}

\begin{frame}{Initial Look at the Data - ATM IV-RV spreads in Asia}


\begin{itemize}
\item Median spreads for S\&P tend to be higher, compared to other liquid global indices, especially at longer horizons (data since 2010)
\end{itemize}


\includegraphics[scale=0.35]{iv_rv_comp_AS}

\end{frame}

\begin{frame}{Initial look at the Data - ATM IV-RV spreads in Europe}

\begin{itemize}
\item Relative to European indices, median SPX spreads also tend to be higher, especially at longer horizons (data since 2010)
\end{itemize}

\includegraphics[scale=0.35]{iv_rv_comp_EU}

\end{frame}

\section{Calendar Collars on International Indices}

\begin{frame}{S\&P 500 Collar, A Recap}

\scalebox{0.65} {

\begin{tabular}{lrrrrrrrrrrr}
\toprule
Strategy (1996-) & $\mu_{arith}$ & TotRetn & $\sigma$ & IR & IR$_{geo}$ & DD & Skew & $\mu_{geo}$ & $\beta$ & $\beta_+$ & $\beta-$ \\
\midrule
Equity & 9.8 & 210.4 & 19.2 & 0.51 & 0.43 & -55.5 & -0.06 & 8.3 & 1.00 & 1.0 & 1.00\\
Equity Collar & 8.5 & 181.5 & 11.1 & 0.76 & 0.74 & -25.4 & -0.33 & 8.2 & 0.47 & 0.4 & 0.45\\
\bottomrule
\end{tabular}


}

\includegraphics[scale=0.35]{spx_collar}
\end{frame}

\begin{frame}{S\&P 500, Collar Since 2010}
Sub-periods are re-runs of simulation, not just a sub-range of the P/L from the previously run simulation
\includegraphics[scale=0.35]{spx_collar_2010}
\end{frame}



\begin{frame}{Eurostoxx Collar}

\scalebox{0.65} {

\begin{tabular}{lrrrrrrrrrrr}
\toprule
Strategy (2002-) & $\mu_{arith}$ & TotRetn & $\sigma$ & IR & IR$_{geo}$ & DD & Skew & $\mu_{geo}$ & $\beta$ & $\beta_+$ & $\beta-$ \\
\midrule
Equity & 6.2 & 97.8 & 23.6 & 0.26 & 0.15 & -57.9 & 0.12 & 3.5 & 1.00 & 1.00 & 1.00\\
Equity Collar & 7.6 & 120.0 & 13.1 & 0.58 & 0.53 & -27.2 & 0.02 & 7.0 & 0.45 & 0.38 & 0.43\\
\bottomrule
\end{tabular}
}

\includegraphics[scale=0.35]{eurostoxx_collar}
\end{frame}

\begin{frame}{Eurostoxx, Collar Since 2010}

\includegraphics[scale=0.35]{sx5e_collar_2010}
\end{frame}




\begin{frame}{Nikkei Collar}

\scalebox{0.65} {

\begin{tabular}{lrrrrrrrrrrr}
\toprule
Strategy (2004/05-) & $\mu_{arith}$ & TotRetn & $\sigma$ & IR & IR$_{geo}$ & DD & Skew & $\mu_{geo}$ & $\beta$ & $\beta_+$ & $\beta-$ \\
\midrule
Equity & 8.9 & 114.5 & 24.2 & 0.37 & 0.25 & -60.4 & -0.29 & 6.2 & 1.00 & 1.00 & 1.0\\
Equity Collar & 9.2 & 118.0 & 39.1 & 0.24 & 0.04 & -66.5 & 0.98 & 1.5 & 0.58 & 0.85 & 0.4\\
\bottomrule
\end{tabular}
}

\includegraphics[scale=0.35]{nikkei_collar}
\end{frame}


\begin{frame}{Nikkei, Collar Since 2010}

\includegraphics[scale=0.35]{nky_collar_2010}
\end{frame}

\begin{frame}{FTSE}

\scalebox{0.65} {

\begin{tabular}{lrrrrrrrrrrr}
\toprule
Strategy (2002-) & $\mu_{arith}$ & TotRetn & $\sigma$ & IR & IR$_{geo}$ & DD & Skew & $\mu_{geo}$ & $\beta$ & $\beta_+$ & $\beta-$ \\
\midrule
Equity & 7.9 & 123.3 & 18.9 & 0.42 & 0.33 & -44.5 & 0.02 & 6.3 & 1.00 & 1.00 & 1.0\\
Equity Collar & 7.8 & 121.6 & 11.3 & 0.69 & 0.66 & -23.6 & -0.29 & 7.5 & 0.51 & 0.44 & 0.5\\
\bottomrule
\end{tabular}

}

\includegraphics[scale=0.35]{ftse_collar}
\end{frame}


\begin{frame}{FTSE, Collar Since 2010}

\includegraphics[scale=0.35]{ftse_collar_2010}
\end{frame}

\section{Enhancements to Calendar Collars}

\begin{frame}{Potential Enhancements to Calendar Collars}

\begin{block}{}
Are there variations of the calendar collar structure that can close the equity return gap while maintaining IR/drawdowns?
\end{block}

\begin{itemize}
\item Currently, we buy a far OTM (-20\%) put. Is it possible to do better by buying protection closer to the money, but vary the notional? 
\item What are the risk/return tradeoffs in targeting $\Delta$'s vs moneyness?
\item Can we enhance returns by selling even shorter maturity calls, to benefit from the more negative $\Theta$
\item Proposal for an enhanced calendar collar structure
\end{itemize}
\end{frame}


\subsection{Moneyness vs Notional}
\begin{frame}{Moneyness vs Notional Tradeoff - S\&P 500}

\begin{itemize}
\item What is our goal? 
	\begin{itemize}
	\item Protect against an instantatenous shock? 
	\begin{itemize}
	\item -10\% days are {\em extremely rare} ($3/22480=0.000133$)  
	\item -5\% days are also quite rare ($75/22480=0.003336$)
	\end{itemize}
	\item Protect against prolonged drawdowns? 
	\begin{itemize} \item Ten $\ge$ -20\%  peak-trough drawdowns since 1920, and two $\ge$ -45\%  since 2000 (a limited sample size)\end{itemize}
	\end{itemize}
\item Stress tests draw our attention to the extremely unlikely case---an instantaneous collapse: protecting against these rare cases can cause significant drag during normal times

\item That said, there {\em are} advantages to buying ATM options: 
\begin{itemize}
\item better liquidity, better spreads and more alternatives to choose from...
\item they are also nice because they remove one key choice---$\Delta$ or moneyness
\end{itemize}

\end{itemize}
\begin{block}{Key Question}
Can we benefit from the better characteristics of ATM options but reducing our costs by varying notionals?
\end{block}

\end{frame}

\begin{frame}{Moneyness vs Notional Tradeoff - S\&P 500 Calendar Collar}
\begin{block}{$\downarrow$ DD $\implies$ $\uparrow$ drag}
\begin{itemize}
\item Eq + short call starts us with a relatively good cost v DD tradeoff
\item If we {\em must} buy a put: $0.2\times$ ATMP $< 1\times$OTM20P$ < 0.4\times$ ATMP
\item Buying anything over $0.4\times$ATMP comes at a high cost
\end{itemize}
\end{block}

\begin{columns}

\begin{column}{0.5\textwidth}

\scalebox{0.7} {

\begin{tabular}{lrrrrr}
\toprule
1996- & $\mu$ & $\sigma$ & $\mu/\sigma$ & DD & Cost \\
\midrule
Equity & 9.8 & 19.2 & 0.51 & -55.5 & 0.0\\
\addlinespace
-1OTM2C...\\
+0.0P & 10.3 & 14.9 & 0.69 & -37.0 & 0.5\\
+0.2ATMP & 9.3 & 12.7 & 0.73 & -29.7 & -0.5\\
+1.0OTM20P & 8.5 & 11.1 & 0.76 & -25.4 & -1.3\\
+0.4ATMP & 8.3 & 10.7 & 0.78 & -24.8 & -1.5\\
+0.6ATMP & 7.4 & 9.0 & 0.82 & -19.7 & -2.4\\
+0.8ATMP & 6.4 & 7.6 & 0.84 & -15.0 & -3.4\\
+1.0ATMP & 5.5 & 7.0 & 0.78 & -10.3 & -4.3\\
\bottomrule
\end{tabular}

}

\end{column}

\begin{column}{0.5\textwidth}

\scalebox{0.7} {


\begin{tabular}{lrrrrr}
\toprule
2010- & $\mu$ & $\sigma$ & $\mu/\sigma$ & DD & Cost\\
\midrule
Equity & 13.6 & 15.2 & 0.90 & -18.5 & 0.0\\
\addlinespace
-1OTM2C...\\
+0.0P & 13.1 & 12.6 & 1.04 & -16.6 & -0.5\\
+0.2ATMP & 11.2 & 10.9 & 1.03 & -14.8 & -2.4\\
+1.0OTM20P & 10.0 & 10.0 & 1.00 & -13.3 & -3.6\\
+0.4ATMP & 9.3 & 9.2 & 1.00 & -13.0 & -4.3\\
+0.6ATMP & 7.4 & 7.7 & 0.96 & -11.2 & -6.2\\
+0.8ATMP & 5.5 & 6.3 & 0.87 & -9.3 & -8.1\\
+1.0ATMP & 3.6 & 5.2 & 0.69 & -7.5 & -10.0\\
\bottomrule
\end{tabular}

}

\end{column}

\end{columns}

\end{frame}


\begin{frame}{Moneyness vs Notional Tradeoff - S\&P 500}

\includegraphics[scale=0.31]{spx_put_notional}
\end{frame}

\subsection{Targeting $\Delta$ vs Moneyness}
\begin{frame}{Targeting $\Delta$'s instead of Moneyness - S\&P500}

\begin{itemize}
\item $\Delta$'s affected by implied volatility
\begin{itemize}
\item targeting $\Delta$'s therefore implies lesser protection while rolling the puts especially when there are IV spikes ($\Delta$'s are high relative to moneyness when IV is high, therefore we effectively buy farther OTM puts)
\item increases peak-to-trough drawdowns 
\end{itemize}
\item However, targeting deltas also mean selling closer to the money calls when IV is very low---increases returns

\end{itemize}


\begin{columns}
\begin{column}{0.5\textwidth}
\scalebox{0.6} {
\begin{tabular}{lrrrrr}
\toprule
1996- & $\mu$ & $\sigma$ & $\mu/\sigma$ & DD & Excess\\
\midrule
Eq & 9.8 & 19.2 & 0.51 & -55.5 & 0.0\\
+11DP,-30DC & 8.5 & 12.1 & 0.70 & -35.5 & -1.3\\
+20OTMP,-2OTMC & 8.5 & 11.1 & 0.76 & -25.4 & -1.3\\
\bottomrule
\end{tabular}
}
\end{column}


\begin{column}{0.5\textwidth}
\scalebox{0.6} {
\begin{tabular}{lrrrrrr}
\toprule
2010- & $\mu$ & $\sigma$ & $\mu/\sigma$  & DD & Excess\\
\midrule
Eq & 13.6 & 15.2 & 0.90 &  -18.5 & 0.0\\
+11DP,-30DC & 10.6 & 10.1 & 1.05 &  -12.7 & -3.0\\
+20OTMP,-2OTMC & 10.0 & 10.0 & 1.00 &  -13.3 & -3.6\\
\bottomrule
\end{tabular}
}
\end{column}
\end{columns}

%\includegraphics[scale=0.34]{spx_collar_delta_v_moneyness}
\end{frame}

\begin{frame}{Targeting $\Delta$'s instead of Moneyness - S\&P500, 2006-}
\includegraphics[scale=0.35]{spx_collar_delta_v_moneyness}
\end{frame}

\begin{frame}{Targeting $\Delta$'s instead of Moneyness - S\&P500, 2010-}


\includegraphics[scale=0.35]{spx_collar_delta_2010}
\end{frame}

\begin{frame}{$\Delta$'s can vary significantly with volatility, unlike moneyness---can affect premiums paid and collected}

\begin{itemize}
\item 20\% 9M OTM puts are about 10D, but in 2008 they were nearly 25D
\end{itemize}

\begin{figure}
\caption{Instrument $\Delta$'s in the simulation}
\includegraphics[scale=0.34]{time_varying_deltas}
\end{figure}

\end{frame}

\begin{frame}{Significant variation in call $\Delta$'s}

\begin{itemize}
\item 1M 2\% OTM calls are less than 20 delta, in 2008 they were over 60D
\end{itemize}

\begin{figure}
\caption{Instrument $\Delta$'s in the simulation}
\includegraphics[scale=0.35]{time_varying_deltas_call}
\end{figure}

\end{frame}

\begin{frame}{Use of $\Delta$'s limits the premium (and possibly protection)}
\begin{figure}
%\caption{Instrument $\Delta$'s in the simulation}
\includegraphics[scale=0.32]{premium_v_iv}
\end{figure}
We will later use this fact to sell moneyness calls and buy $\Delta$ puts
\end{frame}

\subsection{Even Shorter Maturity Calls}
\begin{frame}{Even shorter maturity calls - S\&P 500}

\begin{itemize}
\item Short 2W call narrows  the return gap vs equity at a better IR$\implies$lesser need to lever up 
\end{itemize}

\begin{columns}
\begin{column}{0.7\textwidth}
\includegraphics[scale=0.26]{spx_collar_weekly_call}
\end{column}
\begin{column}{0.3\textwidth} 
    \begin{center}
     \includegraphics[scale=0.31]{spx_collar_short_pl_tx}
     \end{center}
\end{column}
\end{columns}

\end{frame}



\begin{frame}{Even shorter maturity calls - S\&P 500, Since 2010}

\includegraphics[scale=0.37]{spx_collar_weekly_call_2010}

\end{frame}


\section{Summary - {\em Slightly} Lower Protection at a Lower Cost}
\begin{frame}{Enhanced Calendar Collars}

\begin{block}{}
Moving to 10d or 2-week call enhances returns without altering other characteristics. 
Targeting a 11$\Delta$ put, or using an \underline{ATM}  Put at 20\% or 30\% notional are all great candidates.
\end{block}

\begin{figure}

\includegraphics[scale=0.3]{risk_v_return_summary}
\end{figure}
\end{frame}

\begin{frame}{Performance Summary Since 2006 and 2010}


\begin{table}

\centering
\scalebox{0.65}{
\begin{tabular}{lrrrrrr}
\toprule
2006- \footnote{{\tiny All Puts are 270\_60} } & $\mu$ & $\sigma$ & $\mu/\sigma$ & Sortino & DD & Cost \\
\midrule
\checkmark 0.2ATMP-2OTMC\_10\_5 & 10.0 & 13.3 & 0.75 & 0.07 & -29.4 & 0.2\\
Eq  & 9.8 & 19.8 & 0.49 & 0.04 & -55.5 & 0.0\\
\checkmark 11$\Delta$P-2OTMC\_10\_5 & 9.6 & 11.5 & 0.84 & 0.07 & -26.0 & -0.2\\
0.3ATMP-2OTMC\_10\_5 & 9.4 & 12.2 & 0.77 & 0.07 & -26.2 & -0.4\\
11$\Delta$P-2OTMC\_30\_5& 9.0 & 11.2 & 0.80 & 0.07 & -26.3 & -0.8\\
\addlinespace
OTM20P-2OTMC\_10\_5 & 9.0 & 11.1 & 0.81 & 0.07 & -24.3 & -0.8\\
11$\Delta$P-30$\Delta$C\_30\_5 & 8.8 & 11.5 & 0.77 & 0.07 & -35.5 & -1.0\\
0.4ATMP-2OTMC\_10\_5 & 8.8 & 11.1 & 0.79 & 0.07 & -23.7 & -1.0\\
OTM20P-2OTMC\_30\_5 & 8.4 & 10.8 & 0.77 & 0.07 & -25.4 & -1.4\\
\bottomrule
\end{tabular}
}

\end{table}



\begin{table}

\centering

\scalebox{0.64}{
\begin{tabular}{lrrrrrr}
\toprule
2010- & $\mu$ & $\sigma$ & $\mu/\sigma$ & Sortino & DD & Cost \\
\midrule
Eq  & 13.5 & 15.1 & 0.89 & 0.08 & -18.5 & 0.0\\
\checkmark 0.2ATMP-2OTMC\_10\_5 & 11.1 & 11.3 & 0.98 & 0.09 & -14.7 & -2.4\\
11$\Delta$P-30$\Delta$C\_30\_5 & 10.5 & 10.1 & 1.04 & 0.09 & -12.7 & -3.0\\
\checkmark 11$\Delta$P-2OTMC\_10\_5 & 10.3 & 10.7 & 0.97 & 0.09 & -13.2 & -3.2\\
11$\Delta$P-2OTMC\_30\_5 & 10.3 & 10.3 & 1.01 & 0.09 & -13.3 & -3.2\\
\addlinespace
0.3ATMP-2OTMC\_10\_5  & 10.1 & 10.5 & 0.97 & 0.09 & -13.8 & -3.4\\
OTM20P-2OTMC\_10\_5 & 9.9 & 10.5 & 0.95 & 0.08 & -13.2 & -3.6\\
OTM20P-2OTMC\_30\_5 & 9.9 & 10.0 & 0.99 & 0.09 & -13.3 & -3.6\\
0.4ATMP-2OTMC\_10\_5 & 9.2 & 9.7 & 0.95 & 0.08 & -12.9 & -4.3\\
\bottomrule
\end{tabular}
}

\end{table}

\end{frame}

\section{From July 7: Summary and Key Recommendations}

\begin{frame}{A Review of Key Recommendations and Roadmap}

\begin{itemize}
\item Target low $\Delta$ (11) Puts instead of 20\% OTM puts - this prevents us from rolling into expensive insurance in the middle of a stress event
\begin{itemize}
\item Systematic and long-term evidence finds that the use of ATM Puts passively for protection (even at a lower notional) does not offer efficient tradeoffs in terms of their Greek P/L and exposures
\item If we do want to use ATM instruments for other reasons, it's probably fine to allocate 20\% to 30\% to an ATM put instead of a low $\Delta$ Put
\end{itemize}
\item Move towards shorter-term calls (2W) and hold them as long as possible (roll $\le$ 5D before maturity)

\begin{itemize}
\item There is flexibility in the choice of moneyness - farther OTM instruments reduce the need to ``lever'' up the structure to close the equity return gap
\end{itemize}

\item Future
\begin{itemize}
\item Review evidence on the use of levered farther OTM calls for return enhancement
\item Use of calendar collars for Eurostoxx
\item Review evidence on structures like straddles
\end{itemize}

\end{itemize}

\end{frame}


\section{ATM Puts vs Low Delta Puts}

\begin{frame}{Protecting against large losses is far more important than using up insurance for small losses}
\includegraphics[scale=0.38]{compounding}

With that out of the way, we will look at how the calendar collar and variants work during a range of drawdowns, including smaller ones!
\end{frame}

\begin{frame}{Moneyness vs Notional Tradeoff - Review}
\begin{block}{$\downarrow$ DD $\implies$ $\uparrow$ drag}
\begin{itemize}
\item Eq + short call starts us with a relatively good cost v DD tradeoff
\item If we {\em must} buy a put: $0.2\times$ ATMP $< 1\times$OTM20P$ < 0.4\times$ ATMP
\item Buying anything over $0.4\times$ATMP comes at a high cost
\end{itemize}
\end{block}

\begin{columns}

\begin{column}{0.5\textwidth}

\scalebox{0.7} {

\begin{tabular}{lrrrrr}
\toprule
1996- & $\mu$ & $\sigma$ & $\mu/\sigma$ & DD & Cost \\
\midrule
Equity & 9.8 & 19.2 & 0.51 & -55.5 & 0.0\\
\addlinespace
-1OTM2C...\\
+0.0P & 10.3 & 14.9 & 0.69 & -37.0 & 0.5\\
+0.2ATMP & 9.3 & 12.7 & 0.73 & -29.7 & -0.5\\
+1.0OTM20P & 8.5 & 11.1 & 0.76 & -25.4 & -1.3\\
+0.4ATMP & 8.3 & 10.7 & 0.78 & -24.8 & -1.5\\
+0.6ATMP & 7.4 & 9.0 & 0.82 & -19.7 & -2.4\\
+0.8ATMP & 6.4 & 7.6 & 0.84 & -15.0 & -3.4\\
+1.0ATMP & 5.5 & 7.0 & 0.78 & -10.3 & -4.3\\
\bottomrule
\end{tabular}

}

\end{column}

\begin{column}{0.5\textwidth}

\scalebox{0.7} {


\begin{tabular}{lrrrrr}
\toprule
2010- & $\mu$ & $\sigma$ & $\mu/\sigma$ & DD & Cost\\
\midrule
Equity & 13.6 & 15.2 & 0.90 & -18.5 & 0.0\\
\addlinespace
-1OTM2C...\\
+0.0P & 13.1 & 12.6 & 1.04 & -16.6 & -0.5\\
+0.2ATMP & 11.2 & 10.9 & 1.03 & -14.8 & -2.4\\
+1.0OTM20P & 10.0 & 10.0 & 1.00 & -13.3 & -3.6\\
+0.4ATMP & 9.3 & 9.2 & 1.00 & -13.0 & -4.3\\
+0.6ATMP & 7.4 & 7.7 & 0.96 & -11.2 & -6.2\\
+0.8ATMP & 5.5 & 6.3 & 0.87 & -9.3 & -8.1\\
+1.0ATMP & 3.6 & 5.2 & 0.69 & -7.5 & -10.0\\
\bottomrule
\end{tabular}

}

\end{column}

\end{columns}

\end{frame}


\begin{frame}{Do ATM Puts Help with Smaller Drawdowns? {\bf Not Significantly Different}}
\includegraphics[scale=0.35]{atm_puts_dd_2010}
\end{frame}

\begin{frame}{Low $\Delta$ Puts vs ATM Puts: Desired vs Undesired Greeks}

An Option's {\em realized} P/L attribution:

{\small
\begin{multline}
\Pi_{t} \approx \underbrace{\Delta_{t-1}*(S_{t}-S_{t-1}) +\nu_{t-1}*(\sigma_{t}-\sigma_{t-1})  + 0.5*\Gamma_{t-1}*(S_{t}-S_{t-1})^2}_\text{Desired Greeks}  + \\  \underbrace{\textcolor{red}{\Theta_{t-1}}*(\tau_{t}-\tau_{t-1})}_\text{Undesired} + \underbrace{\epsilon_{t}}_\text{b-a spreads,interaction terms, missing greeks {\bf etc} }
\end{multline}
}

\scalebox{0.7}{

\begin{tabular}{l|c|r|r|r|r|r|r|r|r}
\hline
06-17 & Total P/L & $\Delta$ & $\Gamma$ & $\epsilon$ & $\Theta$ & $\nu$ & $\frac{\Delta + \Gamma}{  \left| \Theta  \right|} $ & $\frac{\Delta + \Gamma  + \nu}{  \left| \Theta  \right|} $ & $\frac{\Delta + \Gamma  + \nu+\epsilon}{  \left| \Theta  \right|} $ \\
\hline
\hline
11$\Delta$P & -1.7 & -1.4 & 2.1 & 0.2 & -3.2 & 0.5 & 0.2 & 0.4 & 0.5\\
\hline
ATMP & -6.3 & -4.8 & 4.1 & -0.4 & -5.4 & 0.3 & -0.1 & -0.1 & -0.2\\
\hline
06-09\\
\hline \hline
11$\Delta$P & -0.09 & -0.88 & 4.43 & -0.04 & -3.98 & 0.37 & 0.89 & 0.99 & 0.98\\
\hline
ATMP & -0.34 & -0.85 & 5.76 & -0.50 & -4.83 & 0.07 & 1.02 & 1.03 & 0.93\\
\hline
10-17\\
\hline \hline
11$\Delta$P & -2.53  & -1.62 & 0.95 & 0.30 & -2.80 & 0.64 & -0.24 & -0.01 & 0.10\\
\hline
ATMP & -9.35 & -6.91 & 3.20 & -0.31 & -5.67 & 0.34 & -0.66 & -0.59 & -0.65\\
\hline

\end{tabular}

}

\end{frame}


\begin{frame}{Low $\Delta$ Puts vs ATM Puts: Greek Ratios by Year}
\includegraphics[scale=0.6]{desirable_greeks}
\end{frame}

\begin{frame}{Low $\Delta$ Puts vs ATM Puts: Key Greek P/L by Year}
\includegraphics[scale=0.6]{gamma_vega}
\end{frame}


\begin{frame}{Put Option Characteristics in the Simulation - 07-09 \footnote{In these and the following charts, {\tt OptMV} is the ``market value'' of the option computed as shares $\times$ option price and accounts for the notional allocation. {\bf Greeks are simply instrument greeks and not scaled by the notional } }}

\includegraphics[scale=0.32]{greek_puts_simulation}

\end{frame}

\begin{frame}{Put Instrument Characteristics in the Simulation - 10-17}

\includegraphics[scale=0.32]{greek_puts_simulation_10}

\end{frame}



\begin{frame}{Greek Attribution - Puts}

\includegraphics[scale=0.33]{comparing_the_puts}

\end{frame}


\section{Short-Term Calls and Strikes}
\begin{frame}{Impact of Strikes on Short-Term \footnote{\small all calls are \_10\_5}  Calls}

\begin{columns}

\begin{column}[T]{0.6\textwidth}
\justifying
\scalebox{0.75}{
\begin{tabular}{lrrrrr}
\toprule
2006- & Retn & Risk & IR & Sortino & DD\\
\midrule
Eq & 9.8 & 19.8 & 0.49 & 0.04 & -55.5\\
$11\Delta$ Put -\\
OTM1 & 8.6 & 10.5 & 0.82 & 0.07 & -23.1\\
OTM1.5 & 9.0 & 10.9 & 0.83 & 0.07 & -25.1\\
OTM2.0 & 9.6 & 11.5 & 0.84 & 0.07 & -26.0\\
OTM2.5 & 10.0 & 11.9 & 0.85 & 0.07 & -27.5\\
OTM3.0 & 10.3 & 12.2 & 0.84 & 0.07 & -28.8\\
\bottomrule
\end{tabular}
}

\scalebox{0.75}{
\begin{tabular}{lrrrrr}
\toprule
2010- & Retn & Risk & IR & Sortino & DD\\
\midrule
 Eq & 13.7 & 15.2 & 0.90 & 0.08 & -18.5\\
$11\Delta$ Put -\\
OTM1 
% 
& 9.2 & 9.7 & 0.95 & 0.08 & -12.9\\
OTM1.5 & 9.8 & 10.3 & 0.95 & 0.08 & -13.0\\
OTM2.0 & 10.6 & 10.8 & 0.98 & 0.09 & -13.2\\
OTM2.5 & 11.3 & 11.3 & 1.01 & 0.09 & -13.6\\
OTM3.0 & 11.8 & 11.6 & 1.02 & 0.09 & -13.6\\
\bottomrule
\end{tabular}
}
\end{column}

\hfill


\begin{column}[T]{0.4\textwidth}
\justifying
% \begin{block}{}
\small
\begin{itemize}
\item Selling calls closer to the money provides more protection during volatile periods but also costs more during low vol, positive trend regimes. 
\item Buying farther OTM calls reduces the need to lever up to close the return gap with passive equities. It also keeps $\Delta$'s closer to 0. 
\item Note that selling short-term calls (2W instead of 1M) already keeps deltas lower and closer to 0. 
\end{itemize}
% \end{block}
\end{column}

\end{columns}

\end{frame}


\begin{frame}{Drawdown Profile: Calendar Collar Call Strikes 2010-}

\includegraphics[scale=0.35]{st_call_moneyness_dd_2010}

\end{frame}

\begin{frame}{Drawdown Profile: Calendar Collar Call Strikes 2006-}

\includegraphics[scale=0.35]{st_call_moneyness_dd}

\end{frame}

\begin{frame}{Calendar Collar - $\Delta$ Ranges}
\centering

\scalebox{0.6}{
\begin{tabular}{lrrrrr}
\toprule
06-17 & Median & Max & Min & Mean & Last\\
\midrule
+11$\Delta$P & -0.1127 & -0.0946 & -0.3263 & -0.1210 & -0.1084\\
-OTM1.0C & -0.3254 & -0.0602 & -0.7144 & -0.3327 & -0.0863\\
-OTM1.5C & -0.2301 & -0.0380 & -0.7039 & -0.2578 & -0.0380\\
-OTM2.0C & -0.1485 & -0.0185 & -0.6947 & -0.1931 & -0.0392\\
-OTM2.5C & -0.0940 & -0.0102 & -0.6854 & -0.1460 & -0.0104\\
-OTM3.0C & -0.0595 & -0.0085 & -0.6761 & -0.1132 & -0.0089\\
\bottomrule
\end{tabular}
}

\includegraphics[scale=0.45]{st_call_delta_ranges}
\end{frame}

\begin{frame}{Short Calls and the Greek Tradeoff}

In a short call, we like the $\Delta$ (negative, risk control) and we like the $\Theta$. We \emph{do not} like $\nu$ and being short $\Gamma$. Calls that are farther OTM offer excellent tradeoffs among the desired vs undesired greeks. 

\scalebox{0.85}{
\begin{tabular}{l|r|r|r|r|r|r|r|r}
\hline
06-17 & $\Delta$ & $\Gamma$ & $\epsilon$ & $\Theta$ & $\nu$ & Total & $\frac{\Theta+\Delta}{|\Gamma+\nu|}$ & $\frac{\text{Total}-(\Gamma+\nu)}{|\Gamma+\nu|}$ \\
\hline
OTM1.5 & 1.35 & -11.95 & 2.36 & 13.67 & -4.63 & 0.78 & 0.91 & 1.05\\
\hline
OTM1 & 1.11 & -12.74 & 2.12 & 14.92 & -5.04 & 0.38 & 0.90 & 1.02\\
\hline
OTM2.0 & 1.70 & -10.93 & 2.44 & 12.18 & -4.03 & 1.36 & 0.93 & 1.09\\
\hline
OTM2.5 & 1.92 & -9.83 & 2.55 & 10.72 & -3.53 & 1.83 & 0.95 & 1.14\\
\hline
OTM3.0 & 2.13 & -8.81 & 2.42 & 9.40 & -3.06 & 2.08 & 0.97 & 1.18\\
\hline
\end{tabular}
}
\scalebox{0.85}{
\begin{tabular}{l|r|r|r|r|r|r|r|r}
\hline
10-17 & $\Delta$ & $\Gamma$ & $\epsilon$ & $\Theta$ & $\nu$ & Total & $\frac{\Theta+\Delta}{|\Gamma+\nu|}$ & $\frac{\text{Total}-(\Gamma+\nu)}{|\Gamma+\nu|}$ \\
\hline
OTM1.5 & -1.24 & -9.84 & 2.83 & 11.87 & -4.96 & -1.34 & 0.72 & 0.91\\
\hline
OTM1 & -1.49 & -10.97 & 2.54 & 13.49 & -5.48 & -1.91 & 0.73 & 0.88\\
\hline
OTM2.0 & -0.68 & -8.51 & 2.73 & 10.04 & -4.11 & -0.53 & 0.74 & 0.96\\
\hline
OTM2.5 & -0.17 & -7.20 & 2.60 & 8.38 & -3.43 & 0.18 & 0.77 & 1.02\\
\hline
OTM3.0 & 0.24 & -6.06 & 2.33 & 6.98 & -2.84 & 0.66 & 0.81 & 1.07\\
\hline
\end{tabular}
}

\end{frame}


\begin{frame}{Greeks Attribution of 2W vs 1M Calls}

\includegraphics[scale=0.32]{comparing_the_calls}

\end{frame}


\begin{frame}{Greek Attribution of Short-Term Calls At Different Moneyness}

\includegraphics[scale=0.6]{call_moneyness_greek_attrib}

\end{frame}



\begin{frame}{Call Option Characteristics in the Simulation - 08-09}

\includegraphics[scale=0.32]{greek_calls_simulation_08}

\end{frame}


\begin{frame}{Call Option Characteristics in the Simulation - 2011}

\includegraphics[scale=0.32]{greek_calls_simulation_11}

\end{frame}


\begin{frame}{Call Option Characteristics in the Simulation - 2013}

\includegraphics[scale=0.32]{greek_calls_simulation_13}

\end{frame}


\begin{frame}{Call Option Characteristics in the Simulation - 2015}

\includegraphics[scale=0.32]{greek_calls_simulation_15}

\end{frame}




\begin{frame}{Call Option Characteristics in the Simulation - 2017}

\includegraphics[scale=0.32]{greek_calls_simulation_17}

\end{frame}




\section{Roll Schedule of the Put Leg}

\begin{frame}{Impact of Put  \footnote{Put Maturities, $T=270$} Roll Schedule on the Calendar Collar}


\begin{columns}

\begin{column}[t]{0.6\textwidth}
\justifying
\scalebox{0.8}{

\begin{tabular}{lrrrr}
\toprule
2006- & Retn & Risk & IR & DD\\
\midrule
Eq & 9.8 & 19.8 & 0.49 & -55.5\\
$11\Delta P_{T-180}$ & 8.5 & 11.8 & 0.72 & -30.2\\
$11\Delta P_{T-90}$ & 9.2 & 11.8 & 0.78 & -28.7\\
\checkmark $11\Delta P_{T-60}$ & 9.6 & 11.5 & 0.84 & -26.0\\
$11\Delta P_{T-30}$ & 10.7 & 12.5 & 0.86 & -24.9\\

\bottomrule
\end{tabular}

}

\scalebox{0.8}{

\begin{tabular}{lrrrr}
\toprule
2010- & Retn & Risk & IR & DD\\
\midrule
Eq & 13.7 & 15.2 & 0.90 & -18.5\\
$11\Delta P_{T-180}$  & 9.9 & 10.5 & 0.94 & -14.0\\
$11\Delta P_{T-90}$ & 10.4 & 10.8 & 0.96 & -14.2\\
\checkmark $11\Delta P_{T-60}$ & 10.6 & 10.8 & 0.98 & -13.2\\
$11\Delta P_{T-30}$ & 10.8 & 11.0 & 0.98 & -13.4\\

\bottomrule
\end{tabular}
}
\end{column}

\hfill


\begin{column}[t]{0.4\textwidth}
%\justifying
The roll schedule does not affect risk management. However, frequency of hitting the bid-offer spreads affects returns.\\
 Rolling closer to maturity increases returns at similar risk, at the expense of making outcomes a tad more path dependent.
\end{column}

\end{columns}

\end{frame}

\section{Followup from July 13}

\begin{frame}{Followup from July 13}
\begin{block}{Review}
\begin{itemize}
\item For the short call leg: 2W calls, hold as close as possible to maturity
\item Use 11$\Delta$ puts instead of 20\% OTM; use of $\Delta$'s enforce a modicum of discipline by not buying expensive puts at roll time after a huge sell-off 
\end{itemize}
\end{block}

\begin{block}{Agenda}
\begin{itemize}
\item Show bonds (and other asset returns) in drawdown bar charts
\item Is there a way to increase returns by selectively choosing leverage and moneyness for calls so we can eliminate leverage for the overall structure?
\item Path dependency...
\end{itemize}
\end{block}

\end{frame}

\begin{frame}{Drawdowns and Other Assets}
European and Japanese equities exacerbate drawdowns
\includegraphics[scale=0.33]{dd_plot_assets}
\end{frame}

\begin{frame}{European and Japanese Equities Exacerbate Drawdowns}
\small Unless we have an excellent tactical model, these exposures might partially nullify the benefits of a financially engineered stable equity structure
\includegraphics[scale=0.31]{idxes_dd}
\end{frame}


\begin{frame}{Drawdowns \& the Calendar Collar}
\includegraphics[scale=0.33]{dd_plot_with_bond_1}
\end{frame}


\begin{frame}{Closing the Return Gap by Selling More Calls}
\includegraphics[scale=0.33]{levered_call_2006}
\end{frame}

\begin{frame}{Closing the Return Gap by Selling More Calls: 2010-}
\includegraphics[scale=0.33]{levered_calls_2010}
\end{frame}

\begin{frame}{Selling More Calls - Drawdowns}
Collars start to provide meaningful protection (about 2\%) when equities are down over 10\%
\includegraphics[scale=0.33]{dd_bar_plot_proposal}
\end{frame}

\begin{frame}{Path Dependency}
\small
There is some path dependency depending on how and when you monetize the gains from the Put; they really all even out over time. More importantly, the drawdown protection in all cases is significant
\includegraphics[scale=0.33]{path_dependency}
\end{frame}



\section{Appendix}


\begin{frame}{Sample Call Option Surface\footnote{As of Jun 30}}

\includegraphics[scale=0.32]{sample_call_opt_surface}

\end{frame}

\begin{frame}{Sample Put Option Surface\footnote{As of Jun 30}}

\includegraphics[scale=0.32]{sample_put_opt_surface}

\end{frame}



\end{document}
