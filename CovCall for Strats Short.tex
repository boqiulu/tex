\documentclass{beamer}

% For more themes, color themes and font themes, see:
% http://deic.uab.es/~iblanes/beamer_gallery/index_by_theme.html
%
\mode<presentation>
{
  \usetheme{Madrid}       % or try default, Darmstadt, Warsaw, ...
  \usecolortheme{default} % or try albatross, beaver, crane, ...
  \usefonttheme{professionalfonts}    % or try default, structurebold, ...
  \setbeamertemplate{navigation symbols}{}
  \setbeamertemplate{caption}[numbered]
  \setbeamertemplate{caption}{\insertcaption}
} 

\newcommand{\ditto}[1][.4pt]{\xrfill{#1}~\textquotedbl~\xrfill{#1}}

\usepackage[english]{babel}
\usepackage[utf8x]{inputenc}
\usepackage{graphicx}
\usepackage{booktabs}
\usepackage{paralist}
\usepackage{amsmath}


\AtBeginSection[]
{
  \begin{frame}
    \frametitle{Table of Contents}
    \tableofcontents[currentsection]
  \end{frame}
}


% On Overleaf, these lines give you sharper preview images.
% You might want to `comment them out before you export, though.
\usepackage{pgfpages}

\pgfpagesuselayout{resize to}[letterpaper, landscape] 


% Here's where the presentation starts, with the info for the title slide

\title[Better Strategic $\beta$]{Option Collars for Strategic Equity Allocation in DTR\\Presentation to Strategists}
\author[DC, BL, AV]{Dimitri Curtil, \and Boqiu Lu, \and Anil Vijendran}
\date{\today}

\begin{document}

\begin{frame}
  \titlepage
\end{frame}

\begin{frame}{Enhancing the Strategic Equity Exposure in DTR}
	\begin{itemize}
		\item To deliver equity-like returns, DTR will maintain a core/strategic equity exposure
		\item PROPOSAL: enhance the core\footnote{just US, for now} equity allocation with a collar
			\begin{itemize} \item Passive $\beta$ (futures) $+$ a short position in a short-dated call $+$ a long position in a far out-of-the-money long-dated put \end{itemize}
	
		\item How does it work
			\begin{itemize}
			\item Covered calls reduce $\beta$ while replacing the lower $\beta$ with returns due to the volatility risk premium
			\item Further enhance the strategy with true tail risk protection via a far OTM put
			\item The result is lower vol, drawdowns, and a significant improvement in IR compared to returns from passive equities 
			\item There is catch (always): \emph{lower returns in extreme bull markets}
			\end{itemize}

		\item Rule-based, low-maintenance, relatively easy to implement, rely less on bonds (and on the estimated $\rho_{s/b}$) or active views (for eg security/sector or smart-$\beta$) for downside risk protection
	\end{itemize}

\end{frame}

\begin{frame}{Simulated\footnote{Usual caveats apply...} Performance vs Passive Equities}
Combine the equity leg with 
\begin{itemize} 
\item short call (1m, 2\% OTM), rolled very close to expiry ($T-5$)
\item a long term (9m) 20\% OTM put, rolled considerably before expiry (2m)
\end{itemize}

\scalebox{0.73}{
\begin{tabular}{lrrrrrrrr}
\toprule
Strategy (\textbf{06-16}) & Retn & Geo & Beta & Excess & Risk & IR & IRGeo & MaxDD\\
\midrule
Equity & 9.4 & 7.6 & 1.00 & 0.0 & 20.2 & 0.46 & 0.38 & -55.5\\
Cvd Call, Put & 8.9 & 8.8 & 0.33 & -0.5 & 10.1 & 0.88 & 0.87 & -21.8\\

\end{tabular}
}

\scalebox{0.73}{
\begin{tabular}{lrrrrrrrr}
\toprule
Strategy (\textbf{10-16}) & Retn & Geo & Beta & Excess & Risk & IR & IRGeo & MaxDD\\
\midrule
Equity & 13.4 & 12.9 & 1.00 & 0.0 & 15.6 & 0.86 & 0.83 & -18.5\\
Cvd Call, Put & 9.2 & 9.3 & 0.45 & -4.2 & 8.6 & 1.07 & 1.08 & -9.3\\
\end{tabular}
}
\scalebox{0.73}{
\begin{tabular}{lrrrrrrrr}
\toprule
Strategy (\textbf{2013}) & Retn & Geo & Beta & Excess & Risk & IR & IRGeo & MaxDD\\
\midrule
Equity & 28.6 & 32.3 & 1.00 & 0.0 & 11.1 & 2.58 & 2.92 & -5.8\\
Cvd Call, Put & 23.0 & 25.4 & 0.73 & -5.6 & 8.4 & 2.75 & 3.04 & -3.7\\
\end{tabular}
}
\scalebox{0.73}{
\begin{tabular}{lrrrrrrrr}
\toprule
Strategy (\textbf{14-16}) & Retn & Geo & Beta & Excess & Risk & IR & IRGeo & MaxDD\\
\midrule
Equity & 9.6 & 9.1 & 1.00 & 0.0 & 13.5 & 0.72 & 0.68 & -12.2\\
Cvd Call, Put & 8.0 & 7.9 & 0.62 & -1.6 & 8.8 & 0.91 & 0.90 & -7.1\\
\bottomrule
\end{tabular}
}

\end{frame}


\begin{frame}{1996-2016: Equity-like returns, Lower $\sigma$ and drawdowns}

\includegraphics[scale=0.45]{covcall_call_spread_put_call_dd}

\end{frame}

%\begin{frame}{2010-2016: Better IR in Lower-Vol Periods}
%
%\includegraphics[scale=0.48]{covcall_call_spread_put_call_dd_2010}
%
%\end{frame}
%
%\begin{frame}{Calendar Year Returns 2006-2016}
%
%\includegraphics[scale=0.48]{covcall_collar_bar1}
%
%\end{frame}
%
%
%\begin{frame}{Calendar Year Returns 1996-2005}
%
%\includegraphics[scale=0.48]{covcall_collar_bar2}
%
%\end{frame}
%
%
%\begin{frame}{What to Expect During a Sell-Off}
%\fontsize{6pt}{7.2}\selectfont
%\begin{block}{}
%\begin{itemize}
%
%\item Sell-offs can be prolonged and at volatility levels below 40: 1999-2003, 08 Q1-Q3, or sharp steep drops: 2008 Q4. 
%\item Short-term calls are very good to taking advantage of these prolonged sell-offs at somewhat lower volatility. Puts come in handy for sharp drops (eg 2008). Together, they diversify the ability to offer a better risk-reward trade off.
%\item \textbf{Unless there is a clear short-term view, long puts offer very poor reward for the premium paid}. For example, over the past 20 years a 20\% OTM Put has cost 2.4\% annually. \textbf{$\Delta$-hedging a long put pays up for the VRP without a large portion of the accompanying protection}!
%
%\end{itemize}
%
%\end{block}
%
%\includegraphics[scale=0.35]{sell_off}
%\end{frame}



%
%\begin{frame}{$\Delta$ of the Option Sleeve}
%
%\begin{itemize}
%\item When the instrument is chosen (at roll time, or when the positions are initially established), the median $\Delta$'s of the (short) call and (long) put legs are $\sim -0.3$ and $\sim -0.1$ respectively
%\item However, as can be seen below, there can be considerable variation around the $\Delta$'s of the higher $\gamma$ short-term options, particularly as they approach expiry; while market conditions can change these deltas significantly, routine mechanics as the options approach expiry also affect their $\Delta$'s
%\item We have to be careful as we interpret/explain the $\Delta$-adjusted exposures of our equity positions
%
%\end{itemize}
%\end{frame}
%
%\begin{frame}{$\Delta$ of the Collar - Zooming into 2016}
%\includegraphics[scale=0.48]{covcall_call_deltas_ts}
%%\end{center}
%\end{frame}




\end{document}