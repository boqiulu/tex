\documentclass{beamer}

% For more themes, color themes and font themes, see:
% http://deic.uab.es/~iblanes/beamer_gallery/index_by_theme.html
%
\mode<presentation>
{
  \usetheme{CambridgeUS}       % or try default, Darmstadt, Warsaw, ...
  \usecolortheme{seagull} % or try albatross, beaver, crane, ...
  \usefonttheme{professionalfonts}    % or try default, structurebold, ...
  \setbeamertemplate{navigation symbols}{}
  \setbeamertemplate{caption}[numbered]
  \setbeamertemplate{caption}{\insertcaption}
} 

\newcommand{\ditto}[1][.4pt]{\xrfill{#1}~\textquotedbl~\xrfill{#1}}

\usepackage[english]{babel}
\usepackage[utf8x]{inputenc}
\usepackage{graphicx}
\usepackage{booktabs}
\usepackage{paralist}
\usepackage{amsmath}
\usepackage{color}
\usepackage{ragged2e}


\AtBeginSection[]
{
  \begin{frame}
    \frametitle{Table of Contents}
    \tableofcontents[currentsection]
  \end{frame}
}

% On Overleaf, these lines give you sharper preview images.
% You might want to `comment them out before you export, though.
\usepackage{pgfpages}

\pgfpagesuselayout{resize to}[letterpaper, landscape] 


% Here's where the presentation starts, with the info for the title slide

\title[{\tt GaaOptions}]{{\tt GaaOptions}: Next Steps}
\author[DC $|$ BL]{Dimitri Curtil \and Boqiu Lu}
\date{\today}
%\date{03 November 2017}

\begin{document}

\begin{frame}
  \titlepage
\end{frame}

\section{Variance Risk Premium}
\begin{frame}{Variance Risk Premium}

\begin{itemize}
\item Risk premiums can generally be characterized by the difference between risk-neutral expectations and physical realizations of underlying processes. 
\vfill
\item The {\bf variance risk premium [VRP]} is specified as the difference between implied and realized variance.
\vfill
\item Variance can be a component in determining expected risk and return due to changes in the investment opportunity set, which induces a demand for hedging (Merton 1973); moreover, not only is there variation in returns, but there is also variation in variance. 
\vfill
\item The market rewards bearers of these uncertainties, and while the VRP changes over time, it seems to be {\em different from zero over the long-term}.
\vfill
\end{itemize}
\end{frame}

\begin{frame}{Risk-Neutral Expectation \& Physical Realization}
\begin{itemize}
\item The VRP is directly indicative of the payoff of a variance swap; under no-arbitrage conditions, the variance swap rate equals the risk-neutral expected value of the realized variance, which can be replicated via a portfolio of out-the-money options.
\vfill
\item For global equity indices, the VRP can be obtained from
\begin{itemize}
\item risk-neutral expectation: data from index options [from OptionMetrics]
\item physical realization: data from index (5-minute) intraday returns [from Oxford-Man]
\end{itemize}
\vfill
\end{itemize}
\end{frame}

\begin{frame}{Risk-Neutral Component (1)}
\begin{itemize}
\item Using OptionMetrics data for global index options, for each days, calls and puts are first cleaned and then sorted by maturity and strike price:
	\begin{itemize}
	\item Prices are taken as the average of bid-offer quotes; options with zero bids, bids below [ 3/8 ], and those with quotes that violate common no-arbitrage conditions are discarded.
	\item For any given option, the appropriate interest rate input corresponds to the zero-coupon rate that has a maturity equal to the option's expiration and is obtained by linearly 				interpolating between the two closest zero-coupon rates on the zero curve (following OptionMetrics IvyDB reference manual).
	\end{itemize}
\vfill	
\item To obtain consistent risk-neutral moments, the data is filtered\footnote{\tiny Filtering follows Chang, B., Christoffersen, P., and Jacobs K. {\em Market skewness risk and the cross-section of stock returns.} Journal of Financial Economics. 2013.} to include only:
	\begin{itemize}
	\item Mostly out-the-money contracts (most traded and liquid), i.e. only moneyness levels higher than 97\% for calls and lower than 103\% for puts.
	\end{itemize}
\vfill
\end{itemize}
\end{frame}

\begin{frame}{Risk-Neutral Component (2)}
\begin{itemize}
\item Interpolation / extrapolation:
	\begin{itemize}
	\item Since listed option strikes are discrete, for each date and maturity, implied volatilities are interpolated over a finely-discretized moneyness domain via cubic splice to obtain a dense 			set of implied volatilities.
	\item For out-of-range moneyness levels, implied volatilities are extrapolated from the nearest strike option.
	\item Resulting interpolation-extrapolation generates a fine grid of 1,000 implied volatilities for moneyness levels between 0.01\% and 300\%.
	\end{itemize}
\vfill
\item Pricing: the implied volatilities are then mapped into call and put prices (calls for moneyness levels greater than 1 and puts for moneyness levels less than or equal to 1).
\vfill
\end{itemize}
\end{frame}

\begin{frame}{Physical Component}
\begin{itemize}
\item Measurements of realized variance improve as sampling frequency increases and, in the limit of continuous sampling, converges to the integrated variance; however, in practice, the semi-martingale property (where innovations are orders of magnitude larger than drift for short-term price movements) may be violated at higher than 5-minute frequency due to market microstructure.
\item Using 5-minute intraday returns, aim to obtain an unbiased model-free estimate of the quadratic variation or the integrated variance, $IV_{t}=\int_{0}^{T}\sigma^2_{s}ds$, of the underlying process\footnote{\tiny Liu, L., Patton A., and Sheppard K. {\em Does Anything Beat 5-Minute RV? A Comparison of Realized Variance Measures Across Multiple Asset Classes.} December 2012}.
\item Oxford-Man Institute of Quantitative Finance Realized Library\footnote{\tiny http://realized.oxford-man.ox.ac.uk/} provides daily non-parametric measures of realized variance on benchmark financial assets that are computed using high-frequency data from Thomson Reuters DataScope Tick History database.
\end{itemize}
\end{frame}

\begin{frame}{Global Variance Risk Premiums}
\end{frame}

\section{Preliminary Evaluation: Straddles \& Strangles}
\begin{frame}{Across Indices \& Tenors (Straddles)}
\end{frame}

\begin{frame}{Across Indices \& Deltas (Strangles)}
\end{frame}

\begin{frame}{Across Indices \& Holding Period (Straddles \& Strangles)}
\end{frame}

\section{Option Implied Risk-Neutral Probabilities}
\begin{frame}{Option-Implied Risk-Neutral Probabilities}
\end{frame}
\end{document}